\documentclass{article}
\usepackage{mathtools}


\begin{document}
\title{Homework 1}
\author{Jay Franck}
\maketitle

\section*{3}

Since I found a list of the 25 nearest and 25 brightest stars, I chose to do all 25 of each (excluding white dwarves) instead of the 10 suggested, since the next part of the question asks for 25 anyways.

\includegraphics[scale=0.4]{prob3.eps} 

There is very little overlap between the Brightest Stars and the Nearest stars (a couple of exceptions are Sirius A and Procyon A+B), but otherwise the two sets are very seperate. The nearest stars show a strong linear trend, in that the more blue they are, the brighter they are, and the more red they are, the dimmer they become.

The Brightest stars have only one thing in common: they are all bright. There are bright, red stars that are cool M-type stars (Betelgeuse) and there are hot blue stars as well.

The populations are different because of selection effects. Bright stars can be seen across large distances, and are therefore very easy to pick out from surrounding stars. The nearest stars are part of a volume limited sample, so that the arbitrary variable of distance does not effect the catalogue of stellar types.

\section*{4}


The bolometric correction (calculated in problem 2) did not significantly change the HR diagram. It  shifted all the stars in the plot to a brighter magnitude (the integration of the flux over all wavelengths must be greater than the flux at a single wavelength), but the shape and trend of the populations of stars would not change in this sense.


\includegraphics[scale=0.4]{prob4.eps} 




\end{document}
